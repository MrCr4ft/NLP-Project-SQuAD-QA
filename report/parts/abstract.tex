\chapter*{Abstract} 
    This project aims to study the evolution of population and to assess the emergence of 
    evolutionary stable strategies, in an evolutionary game designed by adding a new strategy, 
    that of the possessor, which embodies the concept of property ownership, to the Hawks-Doves game, 
    an evolutionary model of animal territoriality, in which agents compete for an asset of a certain 
    value.
    The analysis is thoroughly conducted, by simulating several scenarios, with different asset values and 
    fighting costs, probabilities of possessors owning a property, initial conditions, and with or without spatial context.
    It is shown that the dove strategy is not relevant in the H-D-P strategy space, and that, 
    depending on the probability of property ownership and the kind of game (i.e. Prisoner Dilemma or Chicken game),
    different mixed ESS are obtained, with the extreme case of Hawk being the only ESS when no cost is 
    associated with fighting. The size of the neighborhood has an impact on the strategies' distribution, 
    as possessors achieve a higher payoff on average when playing among themselves.
    The conclusion is drawn that ownership conventions and neighborly cooperation tend to 
    naturally emerge from anarchy.